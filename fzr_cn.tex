%!TEX program = xelatex
\documentclass[letterpaper,10pt]{article}

%-----------------------------------------------------------
\usepackage{latexsym}
\usepackage[empty]{fullpage}
\usepackage[usenames,dvipsnames]{color}
\usepackage{verbatim}
\usepackage{hyperref}
\usepackage{framed}
\usepackage{tocloft}
\usepackage{bibentry}

\hypersetup{
    colorlinks,%
    citecolor=black,%
    filecolor=black,%
    linkcolor=black,%
    urlcolor=mygreylink     % can put red here to better visualize the links
}
\urlstyle{same}
\definecolor{mygrey}{gray}{.85}
\definecolor{mygreylink}{gray}{.30}
\textheight=9.0in
\raggedbottom
\raggedright
\setlength{\tabcolsep}{0in}

% Adjust margins
\addtolength{\oddsidemargin}{-0.375in}
\addtolength{\evensidemargin}{0.375in}
\addtolength{\textwidth}{0.5in}
\addtolength{\topmargin}{-.375in}
\addtolength{\textheight}{0.75in}

%-----------------------------------------------------------
%Custom commands
\newcommand{\resitem}[1]{\item #1 \vspace{-2pt}}
\newcommand{\resheading}[1]{{\large \colorbox{mygrey}{\begin{minipage}{\textwidth}{\textbf{#1 \vphantom{p\^{E}}}}\end{minipage}}}}
\newcommand{\ressubheading}[4]{
\begin{tabular*}{6.5in}{l@{\extracolsep{\fill}}r}
		\textbf{#1} & #2 \\
		#3 & #4 \\
\end{tabular*}\vspace{-6pt}}

%Support Chinese
\usepackage{xunicode, xltxtra}
\XeTeXlinebreaklocale "zh"
\usepackage{CJKutf8}
\setmainfont[Mapping=tex-text]{Hiragino Sans GB}
\setsansfont[Mapping=tex-text]{Hiragino Sans GB}
\CJKtilde

%-----------------------------------------------------------

%-----------------------------------------------------------
%General Resume Tips
%   No periods!  Technically, nothing in this document is a full sentence.
%   Use parallelism by ending key words with the same thing,  i.e. "Coordinated; Designed; Communicated".
%   More tips on bottom of this LaTeX document.
%-----------------------------------------------------------

\begin{document}

\newcommand{\mywebheader}{
\begin{tabular*}{7in}{l@{\extracolsep{\fill}}r}
	\textbf{{\Huge 方喆然}} & \href{mailto:i@lazarusx.com}{i@lazarusx.com}\\
     & {手机:+86 152-0192-6975}\\
     & \href{https://github.com/LazarusX}{https://github.com/LazarusX} \\
     & \href{http://stackoverflow.com/users/498996}{http://stackoverflow.com/users/498996} \\
   % Objective: \textbf{Software Engineer} %or \textbf{Data Mining Engineer}
	\end{tabular*}
\\
\vspace{0.1in}}

% CHANGE HEADER SOURCE HERE
\mywebheader
%\resheading{Research Interests}
%	\begin{itemize}
%       \item
%			Data Mining \& Machine Learning
%		\item
%			Propagation in Social Networks
%       \item
%            Algorithms
%	\end{itemize}


%%%%%%%%%%%%%%%%%%%%%%
\resheading{教育}
	\begin{itemize}
       \item
			\ressubheading{{复旦大学}}{}{\emph{计算机软件与理论理学硕士}}{\emph{2013~年~9~月至今}}
			{
				\begin{itemize}
                    
                    \resitem{密码学与信息安全实验室研究助理}
                    \resitem{导师:韩伟力}
                    \resitem{研究生国家奖学金,\emph{总分排名:1/32}}
                    \resitem{研究生新生奖学金一等奖,\emph{总分排名:1/32}}
				\end{itemize}
			}


    \item
			\ressubheading{{复旦大学}}{}{\emph{软件工程工学学士}}{\emph{2009~年~9~月至~2013~年~6~月}}
			{
				\begin{itemize}
                    
                    \resitem{密码学与信息安全实验室研究助理 \emph{(2011~年~2~月至~2013~年~6~月)}}
                    \resitem{导师:韩伟力}
                    \resitem{本科生优秀学生奖学金三等奖}
                    \resitem{本科生优秀学生奖学金二等奖}
				\end{itemize}
			}
	\end{itemize} % End Education list

\resheading{{技能}}
	\begin{itemize}
		\item
            \textbf{编程:} 精通~Java、Android、Objective-C;了解~C、Python、C++
        \item
            \textbf{英语:} CET-6 623~分,流利的英语读写能力,拥有上海英语中级口译证书
            %\textbf{Operating System:} Linux, Windows 
            %\textbf{Tools:} Prezi, R(Self-learning)
        %\ite
             %\textbf{TOEFL:} Reading: 30, Listening: 26, Speaking: 20, Writing: 28, Total: 104 (Nov. 2012)
         %\item \textbf{CET:} CET-4: 623 (Jun. 2010), CET-6: 584 (Dec. 2010)
        % \item    \textbf{GRE:} Verbal: 153 (57\%), Quantitative: 169 (98\%) (Oct. 2012), Analytical Writing: 4.0 (49\%) (Dec. 2011)
	\end{itemize} % End Skills list &

%%%%%%%%%%%%%%%%%%%%%%
\resheading{项目经历}
	\begin{itemize}
		\item
			\ressubheading{Android~感知控制}{科研项目}{Android~框架层、Linux~内核}{\emph{2013~年~9~月至~2014~年~3~月}}
			{\begin{itemize}
				\resitem{开发一个管控~Android~应用程序对感知数据的收集的框架}
                \resitem{改进~Android framework~层的~\texttt{PackageManager}~使得用户能够撤销感知相关的权限}
				\resitem{基于\emph~{SELinux}~实现内核层健壮的防护,以防御恶意的获取~root~权限的企图和绕过~framework~层保护的攻击}
				\resitem{作为国家“核高基”科技重大专项课题的分任务已通过验收并结题}
			\end{itemize}
			}	
	\end{itemize}
			
	\begin{itemize}
		\item
			\ressubheading{浅草网络科技有限公司}{创业项目}{iOS、Android}{\emph{2013~年~11~月至~2014~年~6~月}}
			{\begin{itemize}
				\resitem{参与“大家很忙”(一个分享兴趣和发现活动的社交网络)和“Oran”(一款记录个人每日活动的应用)两款应用的~iOS~客户端开发}
                \resitem{作为项目负责人管理和统筹以上两款应用的~Android~客户端的开发}
			\end{itemize}
			}	
	\end{itemize}
	\begin{itemize}
		\item
			\ressubheading{协同策略管理}{科研项目}{Android、数据挖掘}{\emph{2011~年~3~月至~2014~年~2~月}}
			{\begin{itemize}
				\resitem{提出一种新颖的策略管理框架,核心思想是具有相似功能的应用应该以相似的策略进行管理}
				\resitem{利用基于数据挖掘的方法获取相似的策略,与现有方法相比实现效果显著提高}
				\resitem{基于该项目的一篇~poster~和一篇论文分别被~2011~年~ACM CCS~会议和~IEEE TPDS~期刊录用}
			\end{itemize}
			}	
	\end{itemize}  % End Work Experience list
%%%%%%%%%%%%%%%%%%%%%%

%%%%%%%%%%%%%%%%%%%%%%



%%%%%%%%%%%%%%%%%%%%%%
%\resheading{Academic Experience}
	%\begin{itemize}
		%\item
		%	\ressubheading{Research Assistant}{Fudan University}{Lab of Cryptography and Information Security}{Dec. 2010 -- June. 2012}
		%		{
		%		\begin{itemize}
		%			\resitem{Project: \textbf{Your Password is My Password: An Empirical Study on Cross-Site Password Reuse}. Responsible for the experiment of passwords sharing among different web sites, username and password sharing and password strength analysis. Also involved in the statistics, analysis and writing part of the paper.

					%\resitem{Project: \textbf{Dynamically Combine Authentication Factors based on Quantified Risk and Benefit}}. Responsible for the model modification, experiment interpretation and paper writing.\\

		%		\end{itemize}
		%		}
		%\item
		%	\ressubheading{Research Intern (1 of 4 from Fudan)}{University College Dublin}{Complex \& Adaptive System Lab}{July. 2012 -- Sep. 2012}
		%		{
		%		\begin{itemize}
		%			\resitem{Individual Project: \textbf{Popularity and Sentiment Analysis of Different Mobile Operating Systems on Geo-tagged Tweets}.} 
					%\resitem{Proposed the ideas, identified the brands to compare and designed the structure} 
					%\resitem{Connected MongoDB to store the tweets.Implemented simple NLP methods to analyze the sentiment towards different mobile OSs.} 
					%\resitem{Visualized the data analysis and the sentiment results on global map with Google MAPS APIs.} 
					%\resitem{Prezi Presentation: http://tinyurl.com/qjp9hln}
		%		\end{itemize}
		%		}

 %}

%%%%%%%%%%%%%%%%%%%%%%%%%%%%%%%%	
% \resheading{Course Project}
% \begin{itemize}

% \item \textbf{Indri Search Engine} from \textbf{Search Engine and Web Mining}. Implemented in Java with Indri search engine. Basic search operations of retrieval and also different evaluation methods such as BM25 and tf-idf. The search could work on full text index and achieve desired precision on test.

% \item \textbf{Recommendation System} from \textbf{Search Engine and Web Mining}. Implemented in Java on given authenticate Netflix data sets. With existing user ratings on films, predict the users' ratings on certain films using K-NN method. Achieved the 4th class-wide in blind test.

%\item \textbf{Graph Mining with SQL} from \textbf{Multimedia Databases and Data Mining}. Implemented several Graph mining algorithms such as Six-Degrees, PageRank, Diameter etc on millions of edges and nodes in SQL.

%\item \textbf{Optimization using Decision Trees} from \textbf{Machine Learning}. Implemented in Java with existing Weka libraries. Adjusting the parameters of different decision tree models to optimize the predictions.

%\item \textbf{Question Answering System} from \textbf{Natural Language Processing}. Adopted open source model from Stanford NLP, Indri to create Answering system by processing question, extracting key words and search the answer from background corpus.

% \item \textbf{Migratable Process} from \textbf{Distributed System}. Implemented Migratable Process to allow process runs on one node to migrate to another and continue running with Transactional File I/O.

% \item \textbf{Lunchbox} from \textbf{Mobile Development}. Implemented mobile App of Lunchbox delivery and pickup. Created Android version and backend service for both restaurant side and customer side. Solved the problem for ordering management, order submitting, canceling and lunchbox claiming.

% \item \textbf{Project Table(ongoing)} from \textbf{Capstone Project}. Using Probabilistic Model to learn web page structures and use the learnt structure to extract knowledge and information from web pages, putting them into relational database.

% \end{itemize}
	





%%%%%%%%%%%%%%%%%%%%%%%%%%%%%%%%
\resheading{代表论文}
\begin{itemize}
 \item
 \textbf{Zheran Fang}, Weili Han, Yingjiu Li, \emph{Permission Based Android Security: Issues and Countermeasures}, In Computers \& Security 43 (2014), 205–218 (Dec, 2012)

 \item
 Weili Han, \textbf{Zheran Fang}, Laurence T. Yang, Gang Pan, Zhaohui Wu, \emph{Collaborative Policy Administration}, In IEEE Transactions on Parallel and Distributed Systems
\end{itemize}



%\resheading{{Selected Awards and Honors}}
%\begin{center}
%\begin{tabular*}{6.6in}{l@{\extracolsep{\fill}}r}
        %\multicolumn{2}{c}{TEDxFDU Team Member. Co-host and CTO of TEDxFDUSalon \cftdotfill{\cftdotsep}2012}\\
%        \multicolumn{2}{c}{SCSK Scholarship (1 of 10 in Fudan University) \cftdotfill{\cftdotsep}2012}\\
%		\multicolumn{2}{c}{Morgan Stanley Scholarship (1 of 15 in Fudan University) \cftdotfill{\cftdotsep}2011}\\
		%\multicolumn{2}{c}{Xiyuan Scholarship (Fudan Undergraduate Research Funding) \cftdotfill{\cftdotsep}2011}\\
        %\multicolumn{2}{c}{People Scholarship \cftdotfill{\cftdotsep}2012, 2010}\\
        %\multicolumn{2}{c}{Championship and Best Debater of Fudan Freshman Debating Competition\cftdotfill{\cftdotsep}2009}\\

%		\vphantom{E}
%\end{tabular*}
%\end{center}


\end{document}
